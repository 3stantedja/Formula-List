\documentclass[a4paper]{article}
\usepackage[utf8]{inputenc}
\usepackage{anysize}
\usepackage{amsmath}
\usepackage{mathtools}
% Change fonts here %
%\usepackage{mathpazo}
%\usepackage{mathptmx}
%\usepackage{amsfonts}
\marginsize{2cm}{2cm}{2cm}{2cm}

\title{Equations}
\author{Laurence Amadeus Tristan Tedjapradipta}
\date{April 2018}

\begin{document}

%\maketitle
\begin{center}
    \Large{\textbf{Pure Mathematics}}
\end{center}

\section{Algebra}
\quad \enspace{}For the quadratic equation \(ax^2+bx+c=0\):
\begin{equation}
    x=\frac{-b \pm \sqrt{b^2-4ac}}{2a}
\end{equation}

For an arithmetic series:
\begin{align}
    u_n&=a+(n-1)d,     &  S_n&=\frac{n}{2}(a+l) = \frac{n}{2}\{2a+(n-1)d\}
\end{align}

For a geometric series:
\begin{align}
    u_n&=ar^{n-1}, & S_n&=\frac{a(1-r^n)}{1-r} \ (r \neq 1), & S_\infty &=\frac{a}{1-r} \ (|r| < 1)
\end{align}

Binomial expansion:
\begin{align}
    (a + b)^n = a^n + & \binom{n}{1} a^{n-1} b + \binom{n}{2} a^{n-2} b^2 + \binom{n}{3} a^{n-1} b^3 + \dots + b^n, \text{ where \(n\) is a positive integer} \\
    &\text{and } \binom{n}{r}=\frac{n!}{r!(n-r)!}
\end{align}
\begin{align}
    (1 + x)^n &= 1 + nx + \frac{n(n-1)}{2!} x^2 + \frac{n(n-1)(n-2)}{3!} x^3 + \dots, \text{ where \(n\) is rational and \(|x|<1\)}
\end{align}

\section{Trigonometry}
\begin{equation}
    \text{Arc length of circle} =r\theta \text{ (\(\theta\) in radians)}
\end{equation}
\begin{equation}
    \text{Area of sector of circle} =\frac{1}{2}r^2 \theta \text{ (\(\theta\) in radians)}
\end{equation}

\begin{equation}
    \tan{\theta} \equiv \frac{\sin{\theta}}{\cos{\theta}}
\end{equation}
\begin{align}
    \cos^2{\theta} + \sin^2{\theta} &\equiv 1,  &  1+\tan^2{\theta} &\equiv \sec^2{\theta},  &  \cot^2{\theta}+1 &\equiv \csc^2{\theta}
\end{align}
\begin{align}
    \sin{(A \pm B)} &= \sin A \cos B \pm \cos A \sin B \\*
    \cos{(A \pm B)} &= \cos A \cos B \mp \sin A \sin B \\*
    \tan{(A \pm B)} &= \frac{\tan A \pm \tan B}{1 \mp \tan A \tan B}
\end{align}
\begin{align}
    \sin{2A} &\equiv 2\sin A \cos A \\*
    \cos{2A} &\equiv \cos^2{A} - \sin^2{A} \equiv 2 \cos^2{A}-1 \equiv 1-2\sin^2{A} \\*
    \tan{2A} &\equiv \frac{2 \tan A}{1 - \tan^2{A}}
\end{align}

Principal values:
\begin{gather}
    -\frac{1}{2} \pi \leq \sin^{-1}{x} \leq \frac{1}{2} \pi \\
    0 \leq \cos^{-1}{x} \leq \pi \\
    -\frac{1}{2} \pi \leq \tan^{-1}{x} \leq \frac{1}{2} \pi
\end{gather}

\section{Differentiation}
\begin{align}
    &\mathrm{f}(x)  &  &\mathrm{f}'(x) \nonumber \\
    &x^n            &  &nx^{n-1} \\
    &\ln{x}         &  &\frac{1}{x} \\
    &\mathrm{e}^x   &  &\mathrm{e}^x \\
    &\sin x         &  &\cos x \\
    &\cos x         &  &-\sin x \\
    &\tan x         &  &\sec^2 x \\
    &uv             &  &{ u\frac{\mathrm{d}v}{\mathrm{d}x}+v\frac{\mathrm{d}u}{\mathrm{d}x}} \\
    &\frac{u}{v}    &  &\frac{v\dfrac{\mathrm{d}u}{\mathrm{d}x}-u\dfrac{\mathrm{d}v}{\mathrm{d}x}}{v^2}
\end{align}
If \(x = \mathrm{f}(t)\) and \(y = \mathrm{g}(t)\) then \(\frac{\mathrm{d}y}{\mathrm{d}x} = \frac{\mathrm{d}y}{\mathrm{d}t} \div \frac{\mathrm{d}x}{\mathrm{d}t}\)

\section{Integration}
\begin{align}
    &\mathrm{f}(x)  &  &\text{\(\int\)f}(x) \ \mathrm{d}x \nonumber \\
    &x^n            &  &\frac{x^{n+1}}{n+1} + c \\
    &\frac{1}{x}    &  &\ln|x| + c \\
    &\mathrm{e}^x   &  &\mathrm{e}^x + c \\
    &\sin{x}        &  &-\cos{x} + c \\
    &\cos{x}        &  &\sin{x} + c \\
    &\sec^2 x       &  &\tan{x} + c
\end{align}
\begin{align}
    &\int u \frac{\mathrm{d}v}{\mathrm{d}x} \ \mathrm{d}x = uv - \int v \frac{\mathrm{d}u}{\mathrm{d}x} \ \mathrm{d}x \\
    &\int \frac{\mathrm{f}'(x)}{\mathrm{f}(x)} \ \mathrm{d}x = \ln{|\mathrm{f}(x)|} + c
\end{align}

\section{Vectors}
\quad \enspace{}If \(\mathbf{a} = a_1\mathbf{i} + a_2\mathbf{j} + a_3\mathbf{k}\) and \(\mathbf{b} = b_1\mathbf{i} + b_2\mathbf{j} + b_3\mathbf{k}\) then
\begin{equation}
    \mathbf{a.b} = a_1 b_1 + a_2 b_2 + a_3 b_3 = |\mathbf{a}| |\mathbf{b}| \cos \theta
\end{equation}

\section{Numerical integration}
\quad \enspace{}Trapezium rule:
\begin{equation}
    \int_a^{b} \mathrm{f}(x) \ \mathrm{d}x \approx \frac{1}{2}h\{y_0 + 2(y_1 + y_2 + \dots + y_{n-1}) + y_n\}, \ \text{where \(h = \frac{b-a}{n}\)}
\end{equation}

\end{document}
